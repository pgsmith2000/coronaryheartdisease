\documentclass[]{article}
\usepackage{lmodern}
\usepackage{amssymb,amsmath}
\usepackage{ifxetex,ifluatex}
\usepackage{fixltx2e} % provides \textsubscript
\ifnum 0\ifxetex 1\fi\ifluatex 1\fi=0 % if pdftex
  \usepackage[T1]{fontenc}
  \usepackage[utf8]{inputenc}
\else % if luatex or xelatex
  \ifxetex
    \usepackage{mathspec}
  \else
    \usepackage{fontspec}
  \fi
  \defaultfontfeatures{Ligatures=TeX,Scale=MatchLowercase}
\fi
% use upquote if available, for straight quotes in verbatim environments
\IfFileExists{upquote.sty}{\usepackage{upquote}}{}
% use microtype if available
\IfFileExists{microtype.sty}{%
\usepackage{microtype}
\UseMicrotypeSet[protrusion]{basicmath} % disable protrusion for tt fonts
}{}
\usepackage[margin=1in]{geometry}
\usepackage{hyperref}
\hypersetup{unicode=true,
            pdftitle={Coronary Disease Data Analysis},
            pdfauthor={Paul G. Smith},
            pdfborder={0 0 0},
            breaklinks=true}
\urlstyle{same}  % don't use monospace font for urls
\usepackage{graphicx,grffile}
\makeatletter
\def\maxwidth{\ifdim\Gin@nat@width>\linewidth\linewidth\else\Gin@nat@width\fi}
\def\maxheight{\ifdim\Gin@nat@height>\textheight\textheight\else\Gin@nat@height\fi}
\makeatother
% Scale images if necessary, so that they will not overflow the page
% margins by default, and it is still possible to overwrite the defaults
% using explicit options in \includegraphics[width, height, ...]{}
\setkeys{Gin}{width=\maxwidth,height=\maxheight,keepaspectratio}
\IfFileExists{parskip.sty}{%
\usepackage{parskip}
}{% else
\setlength{\parindent}{0pt}
\setlength{\parskip}{6pt plus 2pt minus 1pt}
}
\setlength{\emergencystretch}{3em}  % prevent overfull lines
\providecommand{\tightlist}{%
  \setlength{\itemsep}{0pt}\setlength{\parskip}{0pt}}
\setcounter{secnumdepth}{0}
% Redefines (sub)paragraphs to behave more like sections
\ifx\paragraph\undefined\else
\let\oldparagraph\paragraph
\renewcommand{\paragraph}[1]{\oldparagraph{#1}\mbox{}}
\fi
\ifx\subparagraph\undefined\else
\let\oldsubparagraph\subparagraph
\renewcommand{\subparagraph}[1]{\oldsubparagraph{#1}\mbox{}}
\fi

%%% Use protect on footnotes to avoid problems with footnotes in titles
\let\rmarkdownfootnote\footnote%
\def\footnote{\protect\rmarkdownfootnote}

%%% Change title format to be more compact
\usepackage{titling}

% Create subtitle command for use in maketitle
\providecommand{\subtitle}[1]{
  \posttitle{
    \begin{center}\large#1\end{center}
    }
}

\setlength{\droptitle}{-2em}

  \title{Coronary Disease Data Analysis}
    \pretitle{\vspace{\droptitle}\centering\huge}
  \posttitle{\par}
    \author{Paul G. Smith}
    \preauthor{\centering\large\emph}
  \postauthor{\par}
      \predate{\centering\large\emph}
  \postdate{\par}
    \date{10/10/2019}


\begin{document}
\maketitle

\hypertarget{coronary-disease-data-analysis}{%
\subsection{Coronary Disease Data
Analysis}\label{coronary-disease-data-analysis}}

This is an example analysis in R to showcase some of the methods and
tools available to analyze a sample dataset. Below is a link to the
heart disease data sets that can be used in this analysis. While each
study contains a broad range of data collected many participants, this
analysis will focus on a core set of 14 variables from one of the
studies (cleveland.data).

This file describes the contents of the heart-disease directory
available at
\url{https:/https://archive.ics.uci.edu/ml/machine-learning-databases/heart-disease/}

This directory contains 4 databases concerning heart disease diagnosis.
All attributes are numeric-valued. The data was collected from the four
following locations:

\begin{verbatim}
 1. Cleveland Clinic Foundation (cleveland.data)
 2. Hungarian Institute of Cardiology, Budapest (hungarian.data)
 3. V.A. Medical Center, Long Beach, CA (long-beach-va.data)
 4. University Hospital, Zurich, Switzerland (switzerland.data)
\end{verbatim}

Each database has the same instance format. While the databases have 76
raw attributes, only 14 of them are actually used. Thus I've taken the
liberty of making 2 copies of each database: one with all the attributes
and 1 with the 14 attributes actually used in past experiments.

The authors of the databases have requested:

\begin{verbatim}
  ...that any publications resulting from the use of the data include the 
  names of the principal investigator responsible for the data collection
  at each institution.  They would be:

   1. Hungarian Institute of Cardiology. Budapest: Andras Janosi, M.D.
   2. University Hospital, Zurich, Switzerland: William Steinbrunn, M.D.
   3. University Hospital, Basel, Switzerland: Matthias Pfisterer, M.D.
   4. V.A. Medical Center, Long Beach and Cleveland Clinic Foundation:
 
\end{verbatim}


\end{document}
